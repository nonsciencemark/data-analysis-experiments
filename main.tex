\documentclass{article}

\usepackage{longtable}
\usepackage{booktabs}
\usepackage{makecell}
\usepackage{outlines}
\usepackage{verbatim}
\usepackage{achemso}
\usepackage{graphicx}
\usepackage{xcolor}
\usepackage{soul}
\usepackage{enumitem}
\usepackage{hyperref}
\usepackage{import}
\usepackage{listings}
\usepackage{authblk}
\usepackage[margin=1in]{geometry}
\usepackage{enumitem}
\usepackage{float}
\usepackage{setspace} \doublespacing
\usepackage{lineno} \linenumbers
\usepackage{amsmath}
\usepackage{bm} % bold maths text
\usepackage{cleveref} % nicer citations

\begin{document}

\title{Environment- and system-specific interactions between traits and population growth}
\author{Mark Holmes, Tessa de Bruin, Nicolas Schtickzelle, Frederik De Laender}

\maketitle

\begin{abstract}

Understanding the factors that influence population dynamics across environmental contexts is essential to predict ecological stability. Functional traits influence population growth, which can in turn influence traits and thus create a feedback between population and trait dynamics. We analysed the reciprocal connections between population and trait dynamics in two microbial systems (one autotrophic and one heterotrophic system) and across a range of environmental conditions. When augmenting models of population (trait) dynamics with trait (population density) data, we found no consistent improvement of predictive capacity: this improvement was environment- and system-dependent. Notably, in the cyanobacteria system, models augmented with population density did predict trait dynamics better. In addition, when augmented models were superior, the improvement was substantial. Further investigation of the environmental dependence of trait-growth relationships is recommended.

\textbf{Keywords:} functional traits, density-dependence, trait-dependence, \dots

\end{abstract}
 % title page is a separate doc bc it gets in the way

\pagebreak

\setlength{\parindent}{0.5cm} % indent first row of paragraph

\section{Introduction}

% we need to predict population dynamics and using traits is good
Forecasting ecological change requires an understanding of population dynamics, the trajectory of population density through time. A prime predictor of a population's growth rate is its current density. Density-dependence of population growth is generally negative, stimulating growth at low density, and limiting it when density gets high, which effectively puts a cap on population size. Since the classic experiments by Gause \cite{Gause1934} and Gill \cite{Gill1972}, evidence of negative density-dependence is abundant across biological systems \cite{Sibly2005}. However, means of functional traits across individuals in a population have also been proposed as essential predictors of population growth \cite{Ozgul2012,Edwards2013,Perez-Ramos2019,Litchman2008,Violle2007a}.

% the augmented pop and trait dynamical system can be complicated and have feedbacks
One challenge when using traits to predict population dynamics is that traits are not static but dynamic. Rapid trait change is widespread \cite{Ellner2011} and, while trait change caused by environmental change is arguably studied more often \cite{Grainger2021}, traits can also change as a result of population density \cite{Gibert2022}. This can happen when population density is strongly-connected to demographic descriptors (e.g. size structure) \cite{Gillooly2001,Brown2004,DeLong2015,Wieczynski2021}. In addition, the trait value itself can be expected to determine trait change, too. This becomes clear when considering body size, often used as a master trait predicting population dynamics \cite{Litchman2008}. Energetic and physiological costs associated with body size limit an individual's growth \cite{Frank2009,Schmidt-Nielsen1984,Haldane1926} in much the same way as population size limits population growth. Two candidate predictors of trait change therefore stand out: population density and the trait value itself. Both population growth and trait change therefore represent a dynamical system of coupled variables. This coupling can create feedbacks, and is therefore important to unravel if we are to predict the fate of populations subject to perturbations. One way to formalize this interdependence of population and trait dynamics is as follows \cite{Patel2018}: 

\begin{equation}
    \begin{split}
        \frac{\text{d}N}{N\text{d}t} & = f(N, T) \\
        \frac{\text{d}T}{\text{d}t} & = g(N, T), \\
    \end{split}
\label{eq:Patel}    
\end{equation}

where $N$ and $T$ are the population density and mean trait value across individuals in a population, and $f$ and $g$ are functions describing the effects of these two predictors on the dynamics of population density and traits. Note that this framework is general; the functions $f$ and $g$ could capture a variety of biological mechanisms, including resource competition, and behavioural, epigenetic, or genetic change. In the context of Eq.~\ref{eq:Patel}, the feedback between population and trait dynamics is captured by the dependence of $f$ on $T$ and of $g$ on $N$. If either dependence is zero, feedback is lost.

% dynammics change between different environments and species
Environmental conditions can affect population growth and traits. For example, increases in temperature favour smaller cell size in plankton \cite{Fernandez-Gonzalez2021}. Furthermore, interactions between growth and traits can be environment-dependent as well, as environmental factors such as temperature have been shown to modulate how traits affect growth \cite{Brown2004, Gillooly2001, Kremer2017a}. One can therefore expect that feedbacks will be environment-dependent, as well as differing among model systems. In systems where the observed traits have a unequivocal contribution to energy or resource acquisition, these relationships are expected to be stronger. A prime example of such systems are primary producers, where traits making up leaf architecture or pigmentation have a direct link to light and carbon fixation and therefore fitness and population growth \cite{Chin2023}. In such systems, trait information may be essential to correctly predict population dynamics \cite{Stomp2004}. How the reciprocal effects (of density on traits) differ between systems is less intuitive, however, and will depend on the capacity and speed of the trait change.

% what do we seek to do
Here, we investigate the reciprocal links between population and trait dynamics in two different microbial systems (ciliates and cyanobacteria) across several environmental conditions. To this end, we performed microcosm experiments where we monitored population densities and trait values over time. Given these data, we first asked whether augmenting a model of density-dependence with traits improved predictions of population growth. Similarly, we then tested whether augmenting a model of trait-dependence with population density improved predictions of trait change. Finally, we asked whether the contributions of traits and density to population and trait dynamics, respectively, depend on the environmental context and/or differ between the two model systems. 

% what did we find broadly %
% We found that traits rarely improve prediction of population dynamics \hl{in any model system}, while the opposite is true for cyanobacteria. The extent to which including traits and density to predict population and trait dynamics, respectively, depended on environmental condition and differed between both systems. 

We found limited evidence to support general feedbacks between population and trait dynamics. Improvements to predictions of these dynamics by including both trait values and population densities was system- and environment-specific. In cases when the augmented model outperformed the single-state model, the prediction increase was substantial. 

\section{Materials and Methods}

\subsection{Experiments}

\subsubsection{Study systems}

% we used cilia and cyano
We addressed these research questions using a number of unicellular, clonally-reproducing strains from two model systems: seven strains from four ciliate genera and four strains from two ecotypes of the picocyanobacteria genus \textit{Synechococcus} \cite{Farrant2016, Grebert2018, Sohm2016}. The ciliate genera represent globally-distributed mixotrophs \cite{Weisse2022,Lu2021} and \textit{Synechoccoccus} makes up a substantial proportion of marine phytoplankton biomass and photosynthesis \cite{Dvorak2014}.

% we looked across a range of conditions
We examined the effects in a range of environmental conditions, represented here by temperature and a pollutant --- both commonly-encountered environmental pressures in natural systems. Specifically, we consider an ``altered'' environment to have increased temperature, increased concentrations of the herbicide atrazine, or both. Temperature change is a ubiquitously-relevant driver to consider as anthropogenically-driven climate change continues to intensify. Atrazine, present in many ecosystems, represents a broad class of herbicides that act via the inhibition of photosynthesis but which also has negative consequences on the fitness of heterotrophic orgnanisms \cite{Shim2022}. Control conditions are hereafter indicated by ``C'', temperature changes by ``T'', atrazine changes by ``A'', and the combined atrazine and temperature changes by ``AT''.

% we monitored growth in monocutlure
We grew the strains in monoculture for sufficient time for them to reach their carrying capacity, enabling us to quantify their intrinsic growth rate and density-dependence. Population densities and trait values were measured at regular, frequent intervals (\cref{tab:treatment_table,sec:sampling_times}).

\begin{table}[H]
    \centering
    \caption{Experimental design of the two model systems, showing how long the experiment lasted, how frequently they were sampled, the number of strains used, the temperature of the warming treatments in degrees Celsius, and the concentration of the pollutant (atrazine). All combinations of pollution and temperature were included in the experiment design. Treatments in bold indicate the reference conditions.}
    \begin{tabular}{|m{2.5cm}|m{2cm}|m{2.5cm}|m{3cm}|m{3cm}|}
        \hline
        Model system & Exp. length (days) & No. of strains & Temperature ($\mathrm{^\circ C}$) & Pollution (atrazine, $\mathrm{mg\,L}^{-1}$) \\
        \hline
        Ciliates & $37$ & $7$ & $\bm{22}, 24$ & $\bm{0}, 20$ \\
        Cyanobacteria & $10$ & $4$ & $\bm{22}, 24$ & $\bm{0}, 0.1$ \\
        \hline
    \end{tabular}
    \label{tab:treatment_table}
\end{table}

\subsubsection{Experimental setup}

% ciliate experiment setup
For the ciliate system, the microcosms consisted of 250 mL square bottles with vented caps (DURAN®) containing 100 mL nutrient medium, inoculated with enough cells to initiate population growth. We obtained the required cell density by a injecting a species-specific volume of 1-week old preculture into new medium (100 \textmu L, 5 mL, 10 mL, and 10 mL for \textit{T. thermophila}, \textit{Loxocephalus} sp, \textit{P. caudatum} and \textit{S. teres}, respectively). The microcosms were kept in an incubator with a 10:14h light/dark cycle. The nutrient medium consisted of Chalkey’s solution supplemented with crushed protist pellets (Carolina biological supply; 0.55 $\mathrm{gL}^{-1}$), filtered through a 0.22 \textmu m filter system (BT50 500 mL) to remove all particulate matter and consecutively bacterized with \textit{Bacillus brevis}, \textit{Brevibacillus subtilis} and \textit{Serratia fonticola}. Following bacterization, the medium was incubated for two days at 20°C on a shaker to allow the bacteria to grow before ciliates were added. During the experiment, we replaced 20\% of the medium every third day by replacing 4 mL culture with 4 mL 5x concentrated unbacterized nutrient medium, supplemented with atrazine as needed.

% cyano experiment setup
For the cyanobacteria system, the microcosms consisted of 6 mL six-well plates (Sarstedt standard flat 6 well plate, 83.3920) of PCRS-11 Red Sea medium \cite{Rippka2000}, inoculated at \~ 5000 cells $\mathrm{mL}^{-1}$ inside temperature-controlled incubators (Lovibond Thermostatic Cabinet). They were cultured in white light (6500K LedAquaristik Sky Bar) in a 12:12h light/dark cycle. The six-well plates were constantly mixed 150 RPM with a VWR mini shaker.

Growth media were adjusted to include the appropriate amount of atrazine and microcosms were kept at a constant temperature throughout. All sampling was undertaken in sterile environments to prevent contamination.

\subsubsection{Sampling procedure}

Population densities and traits were measured using procedures specifically designed for each species: imaging for ciliates and flow cytometry for cyanobacteria.

% ciliate sampling
Ciliate microcosms were sampled by digital darkfield imaging using an much updated version of the method described by  Pennekamp \& Schtickzelle \cite{Pennekamp2013}. For each microcosm, at each sample time, a series of pictures was taken of a 810µL sub-sample using a Sony A9 camera equipped with an FE 90 mm F2.8 Macro G OSS lens and a darkfield approach (indirect light and black background). Each series consisted of 10 second burst shots at 10 fps in grey scale (other settings: 1/160s, F11, ISO 6400), resulting in 100 pictures per sample. We then proceeded with automated image analysis using Fiji \cite{Schindelin2012,VanRossum2022} to count, track, and characterize the traits of all moving particles (i.e. cell size, cell shape, movement speed, and movement linearity) in the sample \cite{Pennekamp2014}. The traits were population averages of cell size, cell aspect ratio, linearity of the cell's movement, and movement speed. Cell morphology influences the maintenance requirement and resource uptake rates and the motility traits affect the encounter rate of ciliates for their bacterial resources. 

% cyano sampling
Cyanobacteria microcosms were sampled using a flow cytometer (Guava easyCyte 12HT) that measures cell densities and trait values using several channels, which correspond to specific excitation and fluorescence combinations by lasers and detectors. The flow cytometer channels correspond approximately to cell size and concentration of three photosynthetic pigments present in \textit{Synechococcus}: chlorophyll-a, phycocyanin, and phycoerythrin. At each sampling time, 57 \textmu L of distilled water was added to counteract evaporation. Then, 200 \textmu L of the 6 mL microcosm was removed and replaced with 200 \textmu L of fresh medium. This 200 \textmu L sample of the culture was diluted with distilled water to be between 50 and 500 cells \textmu $\mathrm{L}^{-1}$, the concentration range for which the flow cytometer gives accurate measurements, and 200 \textmu L of the appropriately-diluted sample was put into a 96-well plate and processed by the flow cytometer. The flow cytometer then measures 1000 cells from each well to determine the cell density and measure the trait values. The traits were population averages of several flow cytometer channels, indicative of cell size and concentrations of chlorophyll-a, phycocyanin, \& phycoerythrin. Cell size again influences resource uptake and maintenance and the photosynthetic pigments influence the light absorption \cite{Maranon2015,Stomp2007a}. Sampled debris, dead cells, and doublets were removed from the analysed data using the CyanoFilter and PeacoQC packages in R version 2.3.1 \cite{Olusoji2021,PeacoQC,RCoreTeam2023}. 

\subsection{Analyses}

The objectives of our analyses were to investigate whether we could best predict the population per-capita growth rate (hereafter ``growth'', $\gamma$) from density only, or if augmenting our prediction using trait data improved this prediction in our model systems. Similarly, we tested if augmenting this model using trait data to predict trait change 
(hereafter $\tau$) with density data improved predictive capacity.

Since there were multiple traits to incorporate, we first summarised the traits into a single aggregate trait (hereafter ``trait'') using a principal component analysis, PCA, per strain and treatment. This eliminates the impact of trait covariance and collinearity, and ensures that trait information is represented by a single variable, as is population density. We use only the first component of this PCA to represent the traits, which contains most of the explained variance (\cref{fig:cilia_PC_var_explained,fig:cyano_PC_var_explained}).

% We computed the probability that the model including both densities and traits as predictors outperformed the single-variable model by computing the Akaike weights \cite{Burnham2004}. 

We calculated, for all combinations of strains and environmental conditions, the growth rate at each time point, $\gamma_t=\Delta^{-1}\textrm{log}(N_{t+\Delta}/N_t)$, where $N$ is population density and $\Delta$ is the time between consecutive sampling times. We then fitted two linear models, for all combinations of strains and environmental conditions, of $\gamma_t$ against $N_t$:

\begin{equation}
\label{eq:pcgr}
\gamma_t = \beta_0 + \beta_1 N_t,
\end{equation}

\begin{equation}
\label{eq:pcgrExtd}
\gamma_t = \beta_0 + \beta_1 N_t + \beta_2 T_t,
\end{equation}

where $N_t$ (density, continuous) and $T_t$ (trait value, continuous) are predictors, and the $\beta_i$ are regression coefficients. The first model, Eq.~\ref{eq:pcgr}, is the standard model of linear, density-dependent growth. The second model, Eq.~\ref{eq:pcgrExtd}, is the model augmented with trait data.

Similarly, for all combinations of strains and environmental conditions, the trait rate of change at each time point $\tau_t$, as $\tau_t=\Delta^{-1}(T_{t+\Delta}-T_{t})$, where $T$ is the trait value, and $\Delta$ is again the time between consecutive sampling times. We again fitted two models, but now using the trait values as the response variables:

\begin{equation}
\label{eq:dT}
\tau_t = \beta_0 + \beta_1 T_t,
\end{equation}

\begin{equation}
\label{eq:dTExtd}
\tau_t = \beta_0 + \beta_1 T_t + \beta_2 N_t,
\end{equation}

where $N_t$ (density, continuous) and $T_t$ (trait value, continuous) are predictors, and the $\beta_i$ are regression coefficients. Again, Eq.~\ref{eq:dTExtd} is the augmented version of Eq.~\ref{eq:dT}. 

With these models at hand, we then tested whether the the augmented model improved predictions of population/trait dynamics by comparing the prediction accuracy of the augmented models relative to the basic models. We compared the AIC of these models to determine which model best predicted the data, transforming these AIC values into Akaike scores representing the probability that the augmented models (\cref{eq:pcgrExtd,eq:dTExtd}) were better than the basic models (\cref{eq:pcgr,eq:dT}) \cite{Wagenmakers2004,Burnham2004}. Finally, in conjunction with the likelihood that the augmented model was best, we compared the relative change in predictive accuracy, relative to the single-variable model, using the coefficient of determination (R-squared). 

\section{Results}

\subsection{Growth and trait change}

\begin{figure}[H]
    \centering
    \includegraphics[width=0.8\linewidth]{figures/cilia/cilia_td_general.pdf}
    \caption{Fitted model parameters showing intrinsic population/trait change (left), density-dependence (centre), and trait dependence (right) in seven strains of ciliates. Large rows indicate whether growth or trait change is being examined and subrows indicate whether the coefficients are from the single-variable or augmented model. Points indicate the mean, error bars indicate the standard error, and asterisks indicate parameters are significantly different from zero (vertical black line).}
    \label{fig:cilia_td_general}
\end{figure}

\begin{figure}[H]
    \centering
    \includegraphics[width=0.8\linewidth]{figures/cyano/cyano_td_general.pdf}
    \caption{As Fig. \ref{fig:cilia_td_general}, but for four strains of the cyanobacteria genus \emph{Synechococcus}.}
    \label{fig:cyano_td_general}
\end{figure}

Populations of both systems exhibited mostly nonzero intrinsic growth and negative density-dependence (\cref{fig:cilia_td_general,fig:cyano_td_general}). Intrinsic growth rates of \textit{Tetrahymena} strains were consistently higher than that of the other strains and \textit{Spirostomum} strains exhibited stronger negative density-dependence (i.e. stronger self-limitation). Environmental conditions had moderate and variable effects on how ciliate populations grew. In the cyanobacteria system, intrinsic growth and density-dependence were comparable across strains (\cref{fig:cyano_pcgr}). However, environmental conditions had greater effects on these parameters; the addition of atrazine substantially reduced intrinsic growth rates (to the point of being negative in cases). 

The traits of both systems exhibited generally negative trait-dependence (\cref{fig:cilia_td_general,fig:cyano_td_general}). In the ciliate system, intrinsic trait change (\cref{fig:cilia_dT}) was generally not significantly different from zero, but estimates varied across environmental conditions in \textit{Tetrahymena} strains. The trait-dependence of trait change in ciliates was also strongest for these strains. In the cyanobacteria system, intrinsic trait change was often significantly different from zero and mostly positive (\cref{fig:cyano_growth_dtrait}).

\subsection{Model selection}

\begin{figure}[H]
    \centering
    \includegraphics[width=\linewidth]{figures/cilia/cilia_AIC.pdf}
    \caption{Differences in model performance when adding traits or density to predict growth (top panel) or trait change (bottom panel) different cilia strains (x-axis) in different treatments (colour). The mean values across strains and treatments are shown by the horizontal dashed lines. Values greater than 0.5 indicate that the full model is likely to be better, values less than 0.5 indicate that the single-variable model is likely to better, and values around 0.5 indicate that the two models are approximately equal.}
    \label{fig:cilia_AIC}
\end{figure}

In the ciliate system, using  the augmented model (including both traits and population densities) to predict growth/trait change generally did not improve predictions (\cref{fig:cilia_AIC_deltaerror}). For predictions of growth, only in a few cases (particularly \textit{Tetrahymena} strains and \textit{Loxocephalus} 1) was the augmented model more likely in some environmental conditions. Predictions of trait change with the augmented model were even less likely to outperform the trait-only model.

\begin{figure}[H]
    \centering
    \includegraphics[width=0.75\linewidth]{figures/cyano/cyano_AIC.pdf}
    \caption{As fig. \ref{fig:cilia_AIC} but for the cyanobacteria system.}
  \label{fig:cyano_AIC}
\end{figure}

In the cyanobacteria system, for treatments without atrazine, the augmented models were generally notably better at predicting the system dynamics (\cref{fig:cyano_AIC}). For predictions of population growth, the augmented model was generally likely to best describe the data, with some exceptions in which the added trait data provided no benefit to predictions. For predictions of trait change, the augmented model was more consistently more likely to be better, with only some cases in the treatments with atrazine where it failed to improve the predictive ability.

\begin{figure}[H]
    \centering
    \includegraphics[width=0.8\linewidth]{figures/cilia/cilia_AIC_deltaerror.pdf}
    \caption{Comparing the change in predictive accuracy using the augmented model (both population and traits) as a response to the probability that the augmented model is best for the ciliate system. Model accuracy was calculated using the R-squared statistic. } 
    \label{fig:cilia_AIC_deltaerror}
\end{figure}

In the ciliate system, the probability that the augmented model fitted best was more beneficial when making predictions of population growth than trait change (\cref{fig:cilia_AIC_deltaerror}). In other words, the cases in which the augmented model was substantially-better fitting (e.g. predictions of growth in \textit{Tetrahymena} 1, control) yielded more accurate predictions of population growth than trait changes.

\begin{figure}[H]
    \centering
    \includegraphics[width=0.8\linewidth]{figures/cyano/cyano_AIC_deltaerror.pdf}
    \caption{As fig. \ref{fig:cilia_AIC_deltaerror} but for the cyanobacteria system.}
  \label{fig:cyano_AIC_deltaerror}
\end{figure}

In the cyanobacteria system, improvement in predictive accuracy by using the augmented model was substantial only when the augmented model was near-certain to be the best (\cref{fig:cyano_AIC_deltaerror}). The augmented model improved predictions of trait change slightly more than predictions of population growth.

We do not observe a consistent, cross-system result as to whether population growth or trait changes are better-predicted by the augmented model. The relationships are clearly environmentally-dependent and system-specific, and variation in augmented model performance is strongly dependent on environmental conditions. That said, for a large number of the cyanobacteria microcosms, particularly in those without atrazine added, the augmented model substantially outperforms the single-variable model (\cref{fig:cyano_AIC_deltaerror}).

\section{Discussion}

We have analysed population and trait dynamics in two microbial systems, across various environmental conditions. Our results do not support a general reciprocal relationship between both types of dynamics, but do indicate that trait dynamics are better predicted when considering population density for the cyanobacteria system, in particular. Furthermore, trait effects on growth are dependent on the environment: a given trait value has differently effects on growth, depending on the (a)biotic conditions (\cref{fig:cilia_td_general,fig:cyano_td_general}) \cite{Gibert2022,Wieczynski2021}.

Broadly-speaking, the population dynamics of the ciliate system across most environmental conditions were better-described by simple density-dependence, as used in most classical population models e.g. Verhulst \cite{Verhulst1838}. Density-dependent growth is clearly a driving force in these cases as their fitted model coefficients were mostly significantly different from zero (\cref{fig:cilia_td_general}) and the augmented model was not strongly supported on the basis of Akaike weights (\cref{fig:cilia_AIC}). Trait dynamics were similarly generally better-described without including population density as a covariate (\cref{fig:cilia_AIC}). While intrinsic trait dynamics are not as well-founded as the equivalent methods in population dynamics (i.e. there are no real equivalents of trait intrinsic growth or carrying capacity), this basic linear trait-dependent trait change still generally outperformed the augmented model. Trait changes in the ciliate system, then, seem unlikely to depend on the population density and, according to these data, feedbacks between populations and traits in the ciliate system are unlikely.

By contrast, in the cyanobacteria system, the augmented model outperformed the population-only model in predicting population dynamics more often than not and, in the cases where it did not fit well, the population-only model was similarly poorly-fitting (\cref{fig:cyano_R2}). Furthermore, trait dynamics in the cyanobacteria system were much more consistently well-described by the augmented model. Particularly in the atrazine-free treatments, where the augmented model was near-certain to always be better, but even in the combined atrazine and temperature (AT) treatments, its performance was overall better. Overall, while population is only modestly trait-dependent, the trait dynamics are clearly influenced by population density. In certain cases, therefore, there is potential for feedbacks between populations and traits (e.g. strain V\_2524 in the control and temperature treatments). 

% Contributions of traits and density to predictive capacity are environment-dependent and differ between the model systems. While, for the ciliate system, we could not identify specific environmental conditions that made said contributions more likely, the control and heated (C and T) treatments in the cyanobacteria system clearly demonstrated trait-dependent growth and population-dependent trait changes. Our data, overall, do not support a strong general feedback between the monitored functional traits and population density, but, particularly in the cyanobacteria system where both populations and traits influences each others dynamics, there is potential.

There are several possible explanations for the patterns observed. Firstly, while we assess the linear relationships between traits and growth rates, it is possible that nonlinear models could more accurately predict the system dynamics. Nonlinear effects of traits on growth are commonplace in natural systems (e.g. the effects of body on growth are generally non-linear \cite{Fernandez-Gonzalez2021}) and such specifics are lost in the simple model formulation here although, for the most part, the linear models used seem appropriate (\cref{fig:cilia_pcgr,fig:cilia_dT,fig:cyano_pcgr,fig:cyano_dT}).

In a similar manner, the better performance of the augmented model in the cyanobacteria system than in the ciliate system may be due to differences in the kind of the traits we measured. The cyanobacteria traits are more directly linked to growth than the ciliate traits. For example, the concentration of photosynthetic pigments within a cell controls absorption of incoming light and therefore growth. Similarly, the effectiveness of light harvesting is likely density-dependent, which makes the pigmentation content subject to plastic change in the time frame of our experiment \cite{Stomp2004}. Indeed, high cell densities cause shading, which can alter pigmentation. Pigmentation, therefore may both influence and depend on population growth \cite{Maranon2015,Stomp2007a}. In contrast, the ciliate traits are as such less-easily linked to growth and thus may require more system-specific analyses. For example, one could summarise the movement and cell size traits into a clearance rate, which may be more strongly-linked to the growth rate of the heterotrophic ciliates \cite{Kiorboe1995,Jacob2019}.

% future work and things we could have done differently
While we have analysed growth curves and trait dynamics to infer coupling between density and traits, akin to time series analysis, a more direct approach would have consisted of a space-for-time approach. In said approach, one experimentally crosses density and trait values, and then monitors short-term growth and trait change (e.g. \cite{Wieczynski2021}). While manipulating density is feasible, manipulating traits is far less so. One approach could be to grow cultures at different densities, which we know will lead to different traits (\cref{fig:cilia_td_general,fig:cyano_td_general}), and then create density gradients by dilution. However, such space-for-time approaches do not guarantee a reliable gradient of functional traits and are more experimentally demanding.

% laying up future papers
We find that the potential for feedbacks is limited in our two experimental systems across the environmental conditions considered. However, in systems with multiple species present (i.e. communities) the potential for feedbacks grows considerably, as indirect interactions may manifest across multiple species' densities and traits (e.g.\cite{Zelnik2022a}). However, one challenge will be that the net result of such feedbacks at the level of population and trait dynamics can be hard to identify, let alone be predicted.

% confessions, maybe remove
Alternative model-fitting approaches such as autoregressive models may also be used to determine the co-dependence of populations and traits. This may improve the analysis in some ways, e.g. by allowing specific error structures that depend on time. Generally, however, such models require longer time series to be effective and are more of use when forecasting time series than in understanding effects of state variables on each other. %These methods, additionally, do not address issue of collinearity between variables (traits) and, if using a summarised trait (e.g. PCA as used here), these models will function similarly to the regressions used.

% conclusion
In conclusion, we found that the relationships between populations and functional traits vary with model system and environmental context. While reciprocal interactions between density and traits were apparent for the cyanobacteria system, we did not find broad support for strong feedbacks between population and trait change. More work is needed to unveil the context dependence of growth-trait connections.

\bibliography{bibliography}

\title{Supplements}

\maketitle

\begin{figure}
    \centering
    \includegraphics[scale=0.8]{figures/cilia/dd_general_cilia.pdf}
    \caption{\hl{need to fix facet labels, hard to see anything in the slope - change scale?} Intrinsic growth rates (left) and density dependence (right) in seven strains of ciliates. Asterisks indicate parameters are significantly different from zero. \hl{not sure why we care if they are different from zero?}}
    \label{fig:dd_general_cilia}
\end{figure}

\begin{figure}
    \centering
    \includegraphics[scale=0.8]{figures/cyano/dd_general_cyano.pdf}
    \caption{\hl{need to fix facet labels} As Fig. \ref{fig:dd_general_cilia}, but for four strains of the cyanobacteria genus \emph{Synechococcus}. Asterisks indicate parameters are significantly different from zero.}
    \label{fig:dd_general_cyano}
\end{figure}
\begin{figure}
    \centering
    \includegraphics[scale=0.8]{figures/cilia/cilia_td_general.pdf}
    \caption{Intrinsic trait change (left) and trait dependence (right) in seven strains of ciliates. Asterisks indicate parameters are significantly different from zero. \hl{again need to update facet labels}}
    \label{fig:td_general_cilia}
\end{figure}

\begin{figure}
    \centering
    \includegraphics[scale=0.8]{figures/cyano/cyano_td_general.pdf}
    \caption{As Fig. \ref{fig:td_general_cilia}, but for four strains of the cyanobacteria genus \emph{Synechococcus}. Asterisks indicate parameters are significantly different from zero. \hl{again need to update facet labels}}
    \label{fig:td_general_cyano}
\end{figure}

\subsection{Ciliate plots}

\subsubsection{Growth}

\begin{figure}[hbt!]
    \centering
    \includegraphics[scale=0.7]{figures/cilia/cilia_pcgr.pdf}
    \caption{Density dependence of the per capita growth rate (pcgr) at various temperature and pesticide treatments, and for seven strains of ciliates.}
    \label{fig:dd_cilia}
\end{figure}

\clearpage

\subsubsection{Traits}

\begin{figure}[hbt!]
    \centering
    \includegraphics[scale=0.7]{figures/cilia/dd_cilia_area.pdf}
    \caption{Density dependence of size at various temperature and pesticide treatments, and for seven strains of ciliates.}
    \label{fig:dd_cilia_area}
\end{figure}

\begin{figure}[hbt!]
    \centering
    \includegraphics[scale=0.7]{figures/cilia/dd_cilia_linearity.pdf}
    \caption{Density dependence of linearity at various temperature and pesticide treatments, and for seven strains of ciliates.}
    \label{fig:dd_cilia_linearity}
\end{figure}

\begin{figure}[hbt!]
    \centering
    \includegraphics[scale=0.7]{figures/cilia/dd_cilia_ar.pdf}
    \caption{Density dependence of aspect ratio at various temperature and pesticide treatments, and for seven strains of ciliates.}
    \label{fig:dd_cilia_ar}
\end{figure}

\begin{figure}[hbt!]
    \centering
    \includegraphics[scale=0.7]{figures/cilia/dd_cilia_speed.pdf}
    \caption{Density dependence of speed at various temperature and pesticide treatments, and for seven strains of ciliates.}
    \label{fig:dd_cilia_speed}
\end{figure}

\clearpage

\subsubsection{Growth vs. traits}

\begin{figure}[hbt!]
    \centering
    \includegraphics[scale=0.7]{figures/cilia/delta_cilia_ar.pdf}
    \caption{Effects on pcgr and on aspect ratio at various temperature and pesticide treatments, and for seven strains of ciliates.}
    \label{fig:delta_cilia_ar}
\end{figure}

\begin{figure}[hbt!]
    \centering
    \includegraphics[scale=0.7]{figures/cilia/delta_cilia_area.pdf}
    \caption{Effects on pcgr and on size at various temperature and pesticide treatments, and for seven strains of ciliates.}
    \label{fig:delta_cilia_area}
\end{figure}

\begin{figure}[hbt!]
    \centering
    \includegraphics[scale=0.7]{figures/cilia/delta_cilia_speed.pdf}
    \caption{Effects on pcgr and on speed at various temperature and pesticide treatments, and for seven strains of ciliates.}
    \label{fig:delta_cilia_speed}
\end{figure}

\begin{figure}[hbt!]
    \centering
    \includegraphics[scale=0.7]{figures/cilia/delta_cilia_linearity.pdf}
    \caption{Effects on pcgr and on linearity at various temperature and pesticide treatments, and for seven strains of ciliates.}
    \label{fig:delta_cilia_linearity}
\end{figure}

% \subsubsection{Causality}

% \begin{figure}[hbt!]
%     \centering
%     \includegraphics[scale=0.7]{figures/cilia/cilia_Granger.pdf}
%     \caption{Granger causality of traits and population for ciliates}
%     \label{fig:cilia_granger}
% \end{figure}


\clearpage

\subsection{Cyanobacteria plots}

\subsubsection{Growth}

\begin{figure}[hbt!]
    \centering
    \includegraphics[scale=0.7]{figures/cyano/dd_cyano.pdf}
    \caption{Density dependence of the per capita growth rate (pcgr) at various temperature and pesticide treatments, and for four strains of cyanobacteria.}
    \label{fig:dd_cyano}
\end{figure}

\clearpage

\subsubsection{Traits}

\begin{figure}[hbt!]
    \centering
    \includegraphics[scale=0.7]{figures/cyano/dd_cyano_FSC.HL.pdf}
    \caption{Density dependence of forward scatter at various temperature and pesticide treatments, and for four strains of cyanobacteria.}
    \label{fig:dd_cyano_FSC.HL}
\end{figure}

\begin{figure}[hbt!]
    \centering
    \includegraphics[scale=0.7]{figures/cyano/dd_cyano_GRN.B.HL.pdf}
    \caption{Density dependence of GRN at various temperature and pesticide treatments, and for four strains of cyanobacteria.}
    \label{fig:dd_cyano_GRN.B.HL}
\end{figure}

\begin{figure}[hbt!]
    \centering
    \includegraphics[scale=0.7]{figures/cyano/dd_cyano_NIR.B.HL.pdf}
    \caption{Density dependence of NIR.B at various temperature and pesticide treatments, and for four strains of cyanobacteria.}
    \label{fig:dd_cyano_NIR.B.HL}
\end{figure}

\begin{figure}[hbt!]
    \centering
    \includegraphics[scale=0.7]{figures/cyano/dd_cyano_NIR.R.HL.pdf}
    \caption{Density dependence of NIR.R at various temperature and pesticide treatments, and for four strains of cyanobacteria.}
    \label{fig:dd_cyano_NIR.R.HL}
\end{figure}

\begin{figure}[hbt!]
    \centering
    \includegraphics[scale=0.7]{figures/cyano/dd_cyano_RED.B.HL.pdf}
    \caption{Density dependence of RED.B at various temperature and pesticide treatments, and for four strains of cyanobacteria.}
    \label{fig:dd_cyano_RED.B.HL}
\end{figure}

\begin{figure}[hbt!]
    \centering
    \includegraphics[scale=0.7]{figures/cyano/dd_cyano_RED.R.HL.pdf}
    \caption{Density dependence of RED.B at various temperature and pesticide treatments, and for four strains of cyanobacteria.}
    \label{fig:dd_cyano_RED.R.HL}
\end{figure}

\begin{figure}[hbt!]
    \centering
    \includegraphics[scale=0.7]{figures/cyano/dd_cyano_SSC.HL.pdf}
    \caption{Density dependence of SSC at various temperature and pesticide treatments, and for four strains of cyanobacteria.}
    \label{fig:dd_cyano_SSC.HL}
\end{figure}

\begin{figure}[hbt!]
    \centering
    \includegraphics[scale=0.7]{figures/cyano/dd_cyano_YEL.B.HL.pdf}
    \caption{Density dependence of YEL.B at various temperature and pesticide treatments, and for four strains of cyanobacteria.}
    \label{fig:dd_cyano_YEL.B.HL}
\end{figure}

\clearpage

\subsubsection{Growth vs. traits}

\begin{figure}[hbt!]
    \centering
    \includegraphics[scale=0.7]{figures/cyano/delta_cyano_FSC.HL.pdf}
    \caption{Effects on pcgr and on FSC at various temperature and pesticide treatments, and for four strains of cyanobacteria.}
    \label{fig:delta_cyano_FSC.HL}
\end{figure}

\begin{figure}[hbt!]
    \centering
    \includegraphics[scale=0.7]{figures/cyano/delta_cyano_GRN.B.HL.pdf}
    \caption{Effects on pcgr and on GRN at various temperature and pesticide treatments, and for four strains of cyanobacteria.}
    \label{fig:delta_cyano_GRN.B.HL}
\end{figure}

\begin{figure}[hbt!]
    \centering
    \includegraphics[scale=0.7]{figures/cyano/delta_cyano_NIR.B.HL.pdf}
    \caption{Effects on pcgr and on NIR.B at various temperature and pesticide treatments, and for four strains of cyanobacteria.}
    \label{fig:delta_cyano_NIR.B.HL}
\end{figure}

\begin{figure}[hbt!]
    \centering
    \includegraphics[scale=0.7]{figures/cyano/delta_cyano_NIR.R.HL.pdf}
    \caption{Effects on pcgr and on NIR.R at various temperature and pesticide treatments, and for four strains of cyanobacteria.}
    \label{fig:delta_cyano_NIR.R.HL}
\end{figure}

\begin{figure}[hbt!]
    \centering
    \includegraphics[scale=0.7]{figures/cyano/delta_cyano_RED.B.HL.pdf}
    \caption{Effects on pcgr and on RED.B at various temperature and pesticide treatments, and for four strains of cyanobacteria.}
    \label{fig:delta_cyano_RED.B.HL}
\end{figure}

\begin{figure}[hbt!]
    \centering
    \includegraphics[scale=0.7]{figures/cyano/delta_cyano_RED.R.HL.pdf}
    \caption{Effects on pcgr and on RED.R at various temperature and pesticide treatments, and for four strains of cyanobacteria.}
    \label{fig:delta_cyano_RED.R.HL}
\end{figure}

\begin{figure}[hbt!]
    \centering
    \includegraphics[scale=0.7]{figures/cyano/delta_cyano_SSC.HL.pdf}
    \caption{Effects on pcgr and on SSC.HL at various temperature and pesticide treatments, and for four strains of cyanobacteria.}
    \label{fig:delta_cyano_SSC.HL}
\end{figure}

\begin{figure}[hbt!]
    \centering
    \includegraphics[scale=0.7]{figures/cyano/delta_cyano_YEL.B.HL.pdf}
    \caption{Effects on pcgr and on YEL.B at various temperature and pesticide treatments, and for four strains of cyanobacteria.}
    \label{fig:delta_cyano_YEL.B.HL}
\end{figure}

\subsubsection{Causality}

% \begin{figure}[hbt!]
%     \centering
%     \includegraphics[scale=0.7]{figures/cyano/cyano_Granger.pdf}
%     \caption{Granger causality of traits and population for cyanobacteria}
%     \label{fig:cyano_grangerl}
% \end{figure}

\clearpage

 %% suppl is a separate doc bc it's enormous

\end{document}
