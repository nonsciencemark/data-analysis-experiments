\section{Deviations from predicted growth vs. deviations from predicted trait values}

The objective of this analysis was to investigate if deviations of pcgr co-occur with deviations of trait values. By ``deviations'', we mean differences between observed pcgr and traits, and the pcgr and traits predicted for the control conditions. In order to calculate deviations on pcgr, we first predicted the pcgr one would expect in absence of treatment effects. To this end, we predicted the pcgr with eq. \ref{eq:pcgr} using the observed densities $N_t$ but with all treatment effects nullified ($\beta_i=0$, $\forall i \neq 0, 3$). We then computed the difference $\delta_\textrm{pcgr}$ between the predicted and observed pcgr. We followed the exact same procedure to calculate the deviations of trait values, i.e. $\delta_\textrm{trait}$ is the difference between the expected (solely predicted from the observed densities, as if no effects would have happened) and the observed trait values. We then regressed $\delta_\textrm{pcgr}$ against $\delta_\textrm{trait}$. Because we make multiple comparisons, we again only concluded an effect was significant when the p-value was $<0.05/(\textrm{nr. of strains}\cdot \textrm{nr. of traits})$. 

Note that deviations of both pcgr and trait values will also occur in the control conditions (model fit to the control data is never perfect). These are the residuals from eq. \ref{eq:pcgr} with all treatment effects nullified ($\beta_i=0$, $\forall i \neq 0, 3$). 