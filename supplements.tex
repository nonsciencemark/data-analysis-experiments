\title{\Huge Supplementary materials}

\maketitle

\setcounter{section}{0}
\setcounter{figure}{0}
\renewcommand{\figurename}{Fig.}
\renewcommand{\thefigure}{S\arabic{figure}}

\section{Sampling regime} \label{sec:sampling_times}

\subsection{Ciliate}

Due to the large differences in cilate population dynamics, the sampling differed between species. Furthermore, since the most important information required to fit growth curves was towards the start of the experiment (in the exponential growth phase), sampling was more frequent at the beginning of the experiment and then declined in frequency as the populations stabilised. \textit{Tetrahhymena} strains were sampled twice a day during the first week at 09:00 and 17:00. All other species were sampled once a day during the first week at 13:00. During the second week, all species were sample once a day at 13:00. During the third week, all species were sampled every second day at 13:00. Finally, during the fourth and fifth weeks, all species were sampled twice a week on Mondays and Thursdays at 13:00.

\subsection{Cyanobacteria}

The cyanobacteria microcosms were sampled every day at the same time for each microcosm: between 09:00 and 13:00 depending on the population, as the sampling procedure took some time. Additionally, each microcosm of the cyanobacteria system had three replicates.

\section{Population density and population change}

\subsection{Ciliate}

\begin{figure}[H]
    \centering
    \includegraphics[width=0.75\linewidth]{figures/cilia/cilia_pcgr.pdf}
    \caption{}
    \label{fig:cilia_pcgr}
\end{figure}

\subsection{Cyanobacteria}

\begin{figure}[H]
    \centering
    \includegraphics[width=0.75\linewidth]{figures/cyano/cyano_pcgr.pdf}
    \caption{}
    \label{fig:cyano_pcgr}
\end{figure}

\section{Trait values and trait change}

\subsection{Ciliate}

\begin{figure}[H]
    \centering
    \includegraphics[width=0.75\linewidth]{figures/cilia/cilia_dT.pdf}
    \caption{}
    \label{fig:cilia_dT}
\end{figure}

\subsubsection{Cyanobacteria}

\begin{figure}[H]
    \centering
    \includegraphics[width=0.75\linewidth]{figures/cyano/cyano_dT.pdf}
    \caption{}
    \label{fig:cyano_dT}
\end{figure}

\section{Principal component variance explained}

\subsection{Ciliate}

\begin{figure}[H]
    \centering
    \includegraphics[width=\linewidth]{figures/cilia/cilia_PC_var_explained.pdf}
    \caption{The explained variance for the first principal component for the cilia models.}
    \label{fig:cilia_PC_var_explained}
\end{figure}

\subsection{Cyanobacteria}

\begin{figure}[H]
    \centering
    \includegraphics[width=\linewidth]{figures/cyano/cyano_PC_var_explained.pdf}
    \caption{The explained variance for the first principal component for the cyanobacteria models.}
    \label{fig:cyano_PC_var_explained}
\end{figure}

\section{Observed vs predicted trait and population change}

\subsection{Ciliate}

\begin{figure}[H]
    \centering
    \includegraphics[width=\linewidth]{figures/cilia/cilia_growth_dtrait.pdf}
    \caption{Observed (x-axis) and predicted (y-axis) population growth (left column) and trait change (left column), including either one (top row) or both predictors (bottom row), for various strains (point types) of ciliates and across environmental conditions (colours).}
    \label{fig:cilia_growth_dtrait}
\end{figure}

\subsection{Cyanobacteria}

\begin{figure}[H]
    \centering
    \includegraphics[width=\linewidth]{figures/cyano/cyano_growth_dtrait.pdf}
    \caption{Observed (x-axis) and predicted (y-axis) population growth (left column) and trait change (left column), including either one (top row) or both predictors (bottom row), for various strains of cyanobacteria (point types) and across environmental conditions (colours).}
    \label{fig:cyano_growth_dtrait}
\end{figure}

\section{Model goodness-of-fit}

\subsection{Ciliate}

\begin{figure}[H]
    \centering
    \includegraphics[width=\linewidth]{figures/cilia/cilia_R2.pdf}
    \caption{The coefficient of determination i.e., r-squared, for the cilia system models. The mean values across strains and treatments are shown by the horizontal dashed lines.}
    \label{fig:cilia_R2}
\end{figure}

\subsection{Cyanobacteria}

\begin{figure}[H]
    \centering
    \includegraphics[width=0.833\linewidth]{figures/cyano/cyano_R2.pdf}
    \caption{As \cref{fig:cilia_R2} but for the cyanobacteria system models.}
    \label{fig:cyano_R2}
\end{figure}

In the ciliate system, the $R^2$ of the model fit was generally below 0.5 for both growth and trait changes for all treatments and was generally unaffected by whether the single- or both-predictor model was used (\cref{fig:cilia_R2}). \textit{Tetrahymena} strains were generally better-fitting but still without substantial model support. In the cyanobacteria system, the treatments without atrazine were substantially better-fitting than those with atrazine added (\cref{fig:cyano_R2}). Additionally, while the goodness-of-fit of the growth predictions were only marginally improved by the addition of the full model, the trait change predictions were greatly improved by including the full model.
